\documentclass[12pt,a4paper]{article}
\usepackage[utf8]{inputenc} %poder escribir tildes directamente
\usepackage[spanish]{babel}
\usepackage[a4paper,top=2.5cm,bottom=2.5cm,left=3cm,right=3cm,marginparwidth=1.75cm]{geometry}
\usepackage{amsmath, amsthm, amsfonts} %% Paquetes de la AMS
\usepackage{mathptmx}%Parecido a New Times Roman
\usepackage{graphicx}
\usepackage[colorlinks=true, allcolors=blue]{hyperref}
\usepackage{caption}									%Figure: texto -> Figure. texto
\captionsetup[figure]{labelsep=period, format=plain}	%Figure: texto -> Figure. texto
\usepackage{listings}
\lstdefinestyle{PythonStyle}{
	language=Python,
	basicstyle=\small\ttfamily,
	commentstyle=\color{magenta},
	keywordstyle=\color{blue},
	stringstyle=\color{red},
	showstringspaces=false, % Evita mostrar los espacios en strings
	numbers=none, %left,
	numberstyle=\tiny\color{blue},
	frame=single,
	breaklines=true,
	breakatwhitespace=true,
	tabsize=4
}
\title{Título}
\author{LILBERICK}
\date{9 de Octubre del 2023}
\begin{document}
\maketitle
\section{INTRODUCCIÓN}
\section{DEFINICIONES}
\subsection{SUBTÍTULO}
\subsubsection{SUBSUBTÍTULO}
El texto lo escribes simplemente así.

Así escribes otro párrafo.

\begin{figure}[h]
	\centering
	\includegraphics[width=0.7\textwidth]{img/1.png}
	\caption{descripcion}
\end{figure}

\begin{lstlisting}[style=PythonStyle]
#codigo python
for i in range(1, 10):
	print(f"Hola Mundo {i}")
\end{lstlisting}

\begin{table}[htbp]
	\begin{center}
		\caption{Geox 8}
		\begin{tabular}{|c|c|}
			\hline
			\textbf{Autonomía} & \textbf{Velocidad}\\
			\hline
			25 minutos & 100 Km/h\\
			\hline
		\end{tabular}
	\end{center}
\end{table}

\begin{center}
	\href{https://www.youtube.com/watch?v=Px4GHgnrj1A}{DescripcionVideo}
\end{center}

\pagebreak
\section*{CONCLUSIONES}
\begin{enumerate}
	\item contenido
\end{enumerate}
\pagebreak
\section*{REFERENCIAS}
\begin{enumerate}
	\item \href{www.google.com}{google}
\end{enumerate}
\end{document}
